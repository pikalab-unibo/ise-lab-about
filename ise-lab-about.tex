%%%%%%%%%%%%%%%%%%%%%%%%%%%%%%%%%%%%%%%%%%%%%%%%%%%%%%%%%%%%%%%%%%%%%%%%%%%%%%%%
% ISE Lab -- About the Lab
% Giovanni Ciatto
% Alma Mater Studiorum - Università di Bologna
% mailto:giovanni.ciatto@unibo.it
%%%%%%%%%%%%%%%%%%%%%%%%%%%%%%%%%%%%%%%%%%%%%%%%%%%%%%%%%%%%%%%%%%%%%%%%%%%%%%%%
%\documentclass[handout]{beamer}\mode<handout>{\usetheme{default}}
%
\documentclass[presentation]{beamer}\mode<presentation>{\usetheme{AMSBolognaFC}}
%\documentclass[handout]{beamer}\mode<handout>{\usetheme{AMSBolognaFC}}
%%%%%%%%%%%%%%%%%%%%%%%%%%%%%%%%%%%%%%%%%%%%%%%%%%%%%%%%%%%%%%%%%%%%%%%%%%%%%%%%
\usepackage{ise-lab-common}
\usepackage{ise-lab-about}
% version
\newcommand{\versionmajor}{1}
\newcommand{\versionminor}{0}
\newcommand{\versionpatch}{1}
\newcommand{\version}{\versionmajor.\versionminor.\versionpatch}
%%%%%%%%%%%%%%%%%%%%%%%%%%%%%%%%%%%%%%%%%%%%%%%%%%%%%%%%%%%%%%%%%%%%%%%%%%%%%%%%
\title[\currentLab{} -- About]{About the Lab}
%
\subtitle{\courseName{} / Module \moduleN{} (\courseAcronym)}
%
\author[\sspeaker{\gcShort}]{\speaker{\gcFull} \\ \gcEmail}
%
\institute[\disiShort, \uniboShort]{\disi{} (\disiShort)\\\unibo}
%
\date[A.Y. \academicYearShort{} (v.\ \version)]{Academic Year \academicYear{}\\(version \version)}
%
%%%%%%%%%%%%%%%%%%%%%%%%%%%%%%%%%%%%%%%%%%%%%%%%%%%%%%%%%%%%%%%%%%%%%%%%%%%%%%%%
\begin{document}
%%%%%%%%%%%%%%%%%%%%%%%%%%%%%%%%%%%%%%%%%%%%%%%%%%%%%%%%%%%%%%%%%%%%%%%%%%%%%%%%

%/////////
\frame{\titlepage}
%/////////

%===============================================================================
\section*{Outline}
%===============================================================================

%/////////
\frame[c]{\tableofcontents[hideallsubsections]}
%/////////

%===============================================================================
\section{Overview}
%===============================================================================

%/////////
\begin{frame}[c,allowframebreaks]{Workflow}

\begin{block}{We address the engineering of \textbf{intelligent systems} (IS)}
	here intended as
	%
	\begin{itemize}
		\item systems composed by one or many \emph{autonomous} and \emph{intelligent} \alert{agents}
		\item situated into some \alert{environment}
		\item pursuing either individual or collective \alert{goals}
	\end{itemize}
\end{block}

\framebreak

``Divide et impera'' approach:
%
\begin{enumerate}
	\item we elicit most fundametal mechanisms of IS
	\item we analyse each aspect individually, discussing:
	\begin{itemize}
		\item the most adequate software \alert{abstractions} supporting them
		\item their engineering w.r.t. software \alert{technologies}
		\item their reification as \alert{code}
	\end{itemize}
	\item we compose them together, to enable (more) complex behaviours
	\item we discuss what we get, and what is missing
	\item we discuss about projects / theses concerning
	\begin{itemize}
		\item topics not fully explored in the course
		\item open problems from the AI literature
		\item lacking technologies from the current state of the art
	\end{itemize}
\end{enumerate}

\end{frame}
%/////////

%===============================================================================
\section{Recall Fundamental Notions}
%===============================================================================

%/////////
\begin{frame}[c]{Disclaimer}
%
\begin{itemize}
	\item We recall a number of notions about IS
	%
	\begin{itemize}
		\item[eg] agents, goals, autonomy, intelligence, the environment, etc. 
	\end{itemize}

	\item Actual definitions are provided by Prof. Omicini in his module
	
	\item Here we just recall the nomenclature and provide insights
	%
	\begin{itemize}
		\item following the purpose of explaining the rationale of our module
	\end{itemize}
\end{itemize}
%
\end{frame}
%/////////

%/////////
\begin{frame}[c]{Agent}
%
\begin{block}{Insight}
	Any entity capable of \alert{acting}\footnote{``acting'' = ``affecting the environment, and (possibly) the agents therein contained''}
	%
	\begin{itemize}
		\item while being situated into some \alert{environment}
		%
		\begin{itemize}
			\item which can be both \alert{perceived} and \alert{affected}
		\end{itemize}

		\item possibly, along with other agents
		%
		\begin{itemize}
			\item with which \alert{interaction} is possible
		\end{itemize}
	\end{itemize}
\end{block}
%
\begin{exampleblock}{Examples of agents}
	\begin{multicols}{2}
		\begin{itemize}
			\item human beings
			\item OS processes
			\item OS threads
			\item logic solvers
			\item robots
			\item BDI agents
		\end{itemize}
	\end{multicols}
\end{exampleblock}
%
\end{frame}
%/////////

%/////////
\begin{frame}[c]{Goal}
%
\begin{block}{Insight}
	A (possibly partial) description of the state of the world\footnote{``world'' $\approx$ ``the environment + other agents} to be reached''
	%
	\begin{itemize}
		\item either by a single agent (\emph{individual} goal)
		\item or by a multi-agent system (\emph{collective} goal)
	\end{itemize}
\end{block}
%
\begin{exampleblock}{Examples of goals}
	\begin{itemize}
		\item vacuum robot $\rightarrow$ ``the floor should be clean''
		\item autonomous car $\rightarrow$ ``reach destination X''
		\item virtual personal assistant $\rightarrow$ ``reminder of meeting, 15 minutes before its start''
	\end{itemize}
\end{exampleblock}
%
\end{frame}
%/////////

%/////////
\begin{frame}[c]{Autonomy (of Agents)}
%
\begin{block}{Insight}
	Agents are \alert{autonomous} when they encapsulate (i.e. control) the criterion
	%
	\begin{itemize}
		\item by which they select which goals to pursue 
		%
		\begin{itemize}
			\item (\emph{motivational} autonomy)
		\end{itemize}
		\item or by which they choose which action to do to while pursuing a goal 
		%
		\begin{itemize}
			\item (\emph{executive} autonomy)
		\end{itemize}
	\end{itemize}
\end{block}
%
\begin{exampleblock}{Examples of autonomous agents}
	\begin{itemize}
		\item human agents are autonomous
		\item software agents may be more or less autonomous 
		%
		\begin{itemize}
			\item depending on how they have been programmed
		\end{itemize}
	\end{itemize}
\end{exampleblock}
%
\end{frame}
%/////////

%/////////
\begin{frame}[c,allowframebreaks]{Intelligence (of Agents)}
%
\begin{block}{Insight}
	Agents are \alert{intelligent} when they have \alert{cognitive capabilities}, and they know when/how to use them to pursue their goal(s)
\end{block}
%
\begin{exampleblock}{Examples of \textbf{cognitive capabilities}}\small
	\begin{itemize}
		\item \alert{perceiving} stimuli and \alert{recognise} abstractions on top of them
		\item \alert{representing knowledge} (e.g. perceptions, abstractions, goals, actions, etc) and \alert{memorising} it for later re-use
		\item \alert{learning} from the experience (i.e. generalise the gathered knowledge)
		\item \alert{planning} courses of action to pursue goals
		\item \alert{reasoning} about knowledge (to \emph{deduce} implicit knowledge, to \emph{induce} new knowledge, to \emph{abduce} hypotheses)
		\item \alert{interact} with other agents to exchange information (goals, knowledge, plans)
		\item etc.
	\end{itemize}
\end{exampleblock}
%
\begin{exampleblock}{Examples of intelligent agents}
	\begin{itemize}
		\item human agents behave intelligently (most of the times)
		\item software agents require cognitive capabilities, to behave intelligently
		%
		\begin{itemize}
			\item plus some criterion to decide when and how to use them
		\end{itemize}
	\end{itemize}
\end{exampleblock}
%
\framebreak
%
\begin{alertblock}{Cognitive capabilities $\nRightarrow$ Intelligence}
	Cognitive behaviours may or may not be considered as intelligent depending on the \alert{context} they are applied into, and on the \alert{observer}
\end{alertblock}
%
\begin{exampleblock}{Examples of intelligent agents}
	\begin{itemize}
		\item agent stepping through the window at ground floor
		\item agent stepping through the window at $N^{th}$ floor
	\end{itemize}
\end{exampleblock}
%
\end{frame}
%/////////

%/////////
\begin{frame}[c]{Environment}
%
\begin{block}{Insight}
	The space where agents live and (inter)act. A.k.a. what is \emph{external} w.r.t. agents.
	%
	\begin{itemize}
		\item enables and constraints agents' \alert{interaction}, \alert{perception}, and \alert{actuation}
	\end{itemize}
\end{block}
%
\begin{exampleblock}{Examples of environments}
	\begin{itemize}
		\item human beings $\rightarrow$ physical world / social media / \ldots
		\item Roomba $\rightarrow$ a house and its floor
		\item chat bot $\rightarrow$ chat groups
		\item autonomous car $\rightarrow$ the road
		\item OS process/thread $\rightarrow$ file system + network + environment variables + I/O
	\end{itemize}
\end{exampleblock}
%
\end{frame}
%/////////

%/////////
\begin{frame}[c]{Perception}
%
\begin{block}{Insight}
	The operation by which agents gather information from the environment
	%
	\begin{itemize}
		\item agents may then \alert{represent}, \alert{memorise}, and \alert{process} perceived information
	\end{itemize}
\end{block}
%
\begin{exampleblock}{Examples of perception}
	\begin{itemize}
		\item human beings $\rightarrow$ 5 sense + introspection + proprioception 
		\item robots $\rightarrow$ input sensors providing raw measurements (cameras, lisars, etc.)
		\item chat bot $\rightarrow$ chat history
		\item OS processes/threads $\rightarrow$ stdin + other input files, environment variables, system clock, network channels, serial ports, etc.
	\end{itemize}
\end{exampleblock}
%
\end{frame}
%/////////

%/////////
\begin{frame}[c]{Actuation}
%
\begin{block}{Insight}
	The operation by which agents affect the environment
	%
	\begin{itemize}
		\item agents may be provided with number of elementary actions
		%
		\begin{itemize}
			\item or be capable of learning them along the way
		\end{itemize}
		\item to be combined in infinitely many ways
	\end{itemize}
\end{block}
%
\begin{exampleblock}{Examples of actuation}
	\begin{itemize}
		\item human beings $\rightarrow$ hands, feets, virtually any limb of our bodies, speech
		\item robots $\rightarrow$ actuators (wheels, arms, leds, etc.) 
		\item chat bot $\rightarrow$ sending messages
		\item OS processes/threads $\rightarrow$ stdout + other output files, environment variables, network channels, serial ports, etc.
	\end{itemize}
\end{exampleblock}
%
\end{frame}
%/////////

%/////////
\begin{frame}[c]{Interaction}
%
\begin{block}{Insight}
	Where agents affect (and are affected by) each others
	%
	\begin{itemize}
		\item may involve both perception and actuation
		\item does not strictly imply \emph{communication}
	\end{itemize}
\end{block}
%
\begin{exampleblock}{Examples of actuation}
	\begin{itemize}
		\item human beings $\rightarrow$ speech, mails, chats, non-verbal communication, etc.
		\item robots $\rightarrow$ stigmergy, mutual perception, \ldots
		\item chat bot $\rightarrow$ messages with buttons
		\item OS processes/threads $\rightarrow$ message passing, tuple spaces/centres, etc.
	\end{itemize}
\end{exampleblock}
%
\end{frame}
%/////////

%===============================================================================
\section{About the Lab}
%===============================================================================

%/////////
\begin{frame}[c,allowframebreaks]{Fundamental Mechanisms of Intelligent Systems}
%
\begin{enumerate}
	\item Knowledge representation 
	%
	\begin{itemize}
		\item How is information represented to favour memorisation and processing
	\end{itemize}
	%
	\item Inference 
	%
	\begin{itemize}
		\item How novel/explicit knowledge is attained from prior/implicit knowledge
	\end{itemize}
	%
	\item Perception / actuation 
	%
	\begin{itemize}
		\item How information exchange with the environment occurs
	\end{itemize}
	%
	\item Planning 
	%
	\begin{itemize}
		\item How courses of actions (directed towards some goal) are computed
	\end{itemize}
	%
	\item Learning 
	%
	\begin{itemize}
		\item How experience is transformed into knowledge
	\end{itemize}
	%
	\item Deliberation 
	%
	\begin{itemize}
		\item Where all such aspects are tied together
	\end{itemize}
	%
	\item Environment and artefacts 
	%
	\begin{itemize}
		\item Where (inter)actions are coordinated
	\end{itemize}
	%
\end{enumerate}
%
\framebreak
%
\begin{block}{Special guest: \textbf{Explanation}}
	How knoweldge is \alert{transferred} from an agent to another
	%
	\begin{itemize}
		\item software to human agent $\leftarrow$ symbolic knowledge \alert{extraction}
		\item human to software agent $\leftarrow$ symbolic knowledge \alert{injection}
	\end{itemize}
\end{block}
%
\begin{block}{Special guest: \textbf{Argumentation}}
	Letting agents reach shared agreements by \alert{arguing}
	%
	\begin{itemize}
		\item reason about how arguments attack each other
		\item reason about when to present an argument to the opponent(s)
	\end{itemize}
\end{block}
%
\end{frame}
%/////////

%/////////
\begin{frame}[c,allowframebreaks]{Orthogonal topics}

Topics necessary to understand two or more mechanisms:

\begin{block}{Data science (a.k.a. machine learning, a.k.a. data mining)}
	The use of statistics and algorithms to extract knowledge out of data
	%
	\begin{itemize}
		\item necessary or useful for: 
		\vspace{-10pt}
		\begin{multicols}{2}
			\begin{itemize}
				\item perception
				\item learning
				\item inference
				\item processing
			\end{itemize}
		\end{multicols}
	\end{itemize}
\end{block}
%
\begin{itemize}
	\item we assume students already have a background on this topic
	%
	\begin{itemize}
		\item yet, most relevant aspects may be recalled when needed
	\end{itemize}
\end{itemize}

\begin{block}{Computational Logic}
	The use of logic to \emph{perform} or \emph{reason about} computation
	%
	\begin{itemize}
		\item necessary or useful for: 
		\vspace{-10pt}
		\begin{multicols}{2}
			\begin{itemize}
				\item knowledge representation
				\item reasoning
				\item planning
				\item learning
				\item perception
				\item deliberation
			\end{itemize}
		\end{multicols}
	\end{itemize}
\end{block}
%
\begin{itemize}
	\item we provide background on this topic, along the course
\end{itemize}

\end{frame}
%/////////

\subsection{Required Technologies and Skills}

\begin{frame}[c,allowframebreaks]{Required Technologies and Skills}

\begin{block}{Legend}
	\begin{multicols}{2}
		\begin{itemize}
			\item[$\checkmark$] we assume you know this topic
			\item[$\rightarrow$] we teach this topic
		\end{itemize}
	\end{multicols}
\end{block}

\framebreak

\begin{alertblock}{Required}
	\begin{itemize}
		\item[$\checkmark$] distributed version control systems (DVCS) \& \alert{Git}
		%
		\begin{itemize}
			\item useful resources: \ccite{pianiniDvcs, proGit}
		\end{itemize}

		\vfill

		\item[$\checkmark$] build automation tools and \alert{Gradle}
		%
		\begin{itemize}
			\item useful resources: \ccite{pianiniBuildAutomation, gradleUserGuide}
		\end{itemize}
		
		\vfill

		\item[$\checkmark$] OO programming on the JVM (11+)
		%
		\begin{itemize}
			\item useful resources: \ccite{ViroliOOP2122,Hunt21} %ProgrammizJavaIO,Naftalin2006,Bloch2008, JavaCollectionsCheatsheets,Warburton2014, Lea1999,Oaks2004, Garg2004,Goetz2006}
			\item admissible languages: Java, \alert{Kotlin}
		\end{itemize}

		\vfill

		\item[$\rightarrow$] logic programming in Prolog
		%
		\begin{itemize}
			\item useful resources: \ccite{SterlingS94,2pkt-swx16}
			\item admissible implementations: \alert{\twopkt{}}, SWI-Prolog
		\end{itemize}

	\end{itemize}
\end{alertblock}

\begin{exampleblock}{Useful}
	\begin{itemize}
		\item[$\checkmark$] basic shell scripting and \alert{Bash}
		\item[$\checkmark$] basic IDE configuration and usage (\alert{Eclipse}, \alert{IntelliJ Idea}, or whatever)
		%
		\begin{itemize}
			\item choice, configuration, and usage of your environment is up to you
		\end{itemize} 
		\item[$\checkmark$] programming in Python (3.9.x)
		%
		\begin{itemize}
			\item useful resources: \ccite{Shaw2017, scikit-learn, psyke-woa2021}
			\item useful libraries: SciKit-learn, 2ppy, Psyke
			\item useful tools: Pip, Virtualenv, Pyenv
		\end{itemize}
	\end{itemize}
\end{exampleblock}

\end{frame}

\section{Conventions and Suggestions}

\begin{frame}[c,allowframebreaks]{About Lab Lectures}

    \begin{itemize}

        \item Lab lectures may involve one or more practical exercises
        %
        \begin{itemize}
            \item commonly started during the lab lecture
            \item rarely completed within the same lecture
            %
			\begin{itemize}
				\item[!] when this is the case, you are supposed to \emph{autonomously} complete the exercise(s), \emph{eventually}
			\end{itemize}
        \end{itemize}

        \bigskip

        \item You are supposed to learn something, not just solving puzzles
        %
        \begin{itemize}
            \item no deadline, no mark
            \item[$\rightarrow$] take your time
            \item[$\rightarrow$] ask for help on the forum
            \item[$\rightarrow$] compare/discuss with your colleagues
        \end{itemize}

        \bigskip

        \item \alert{All} exercises are \alert{optional}
        %
        \begin{itemize}
			\item `optional' = `their completion is not explictly checked before the exam'
            \item solutions and comments are provided \emph{upon request}
        \end{itemize}

        \framebreak

        \item Yet, of Lab-related topics provide useful background knowledge for your project activities

        \bigskip

        \item Exercises are provided as GitLab repositories hosting software projects
        %
        \begin{itemize}
			\item build automation tools are provided to guarantee reproducibility
            \item you need a GitLab account $\rightarrow$ follow the instructions described on slide \ref{slide:submissions}
        \end{itemize}

        \bigskip

        \item Slides are provided as \emph{versioned} PDF, through GitHub releases
        %
        \begin{itemize}
            \item[eg] \uurl{https://github.com/pikalab-unibo/ise-lab-about/releases}
        \end{itemize}

    \end{itemize}

\end{frame}

\begin{frame}[c,allowframebreaks]{About Exercises Submissions}
\label{slide:submissions}

    \begin{itemize}
		\item Submission are optional!
		%
        \begin{itemize}
            \item yet comments and solutions are \emph{only} provided via GitLab
            \item please avoid snippets or zips containing code
        \end{itemize}

		\bigskip

        \item Your code should be pushed on the same GitLab repository exercises have been cloned from, \alert{through Git}

        \bigskip

        \item To this end, you should create a GitLab account %, following instructions provided in \cite{envSetup}:
        %
        \begin{itemize}
            \item possibly, using your institutional credentials \texttt{\alert{name.surnameN}@studio.unibo.it}
            \item possibly, using \texttt{\alert{name.surnameN}} as username
        \end{itemize}

        \framebreak

        \item After that, you are supposed to request access on the \courseAcronym{} \academicYearShort{} GitLab group:
        %
        \begin{itemize}
            \item \url{\courseGroup}
        \end{itemize}

        \bigskip

        \item Submissions should be pushed on a branch named \alert{\texttt{submissions/\textit{name.surnameN}}} %, as described in \cite{envSetup}
        %
        \begin{itemize}
            \item one solution per student ($\implies$ no group submissions)
            \item solutions provided elsewhere will be ignored
        \end{itemize}
    \end{itemize}
\end{frame}

\begin{frame}[c]{About Fora}
    \begin{itemize}
		\item Use the \emph{general} forum as much as possible
		\\
		\uurl{\generalForum}
		%
		\begin{itemize}
			\item don't be shy :)
			\item ask for help if you need it
			\item ask why if you are curious
			\item compare you solutions
			\item be critical and provide suggestions if you feel so
		\end{itemize}

		\bigskip

		\item Requests for advices, comments, solutions, as well as critics or issues should be reported on the \emph{general} forum \alert{only}
		%
		\begin{itemize}
			\item if sent by email, you will be requested to post them on the general forum
		\end{itemize}

		\bigskip

		\item Privacy-sensitive topics can and should be discussed via email
	\end{itemize}
\end{frame}

\section{The Final Project}

\begin{frame}[c]{About Projects}
    \begin{itemize}
        \item detailed rules here
		\\
        \uurl{\projectRules}

        \vfill

        \item workflow overview
        %
        \begin{enumerate}
            \item \alert{choose} a project or \alert{propose} one
            \item reserve your project on the \href{\projectsForum}{Projects forum}
            \item submit an \alert{initial report}, describing your own requirements
            \item receive a Git repository for tracking the development of your artefacts
            \item develop your projects
            \item write the \alert{final report}
            \item submit your project's \alert{report} on the \href{\projectsForum}{Projects forum}
            %
			\begin{itemize}
				\item push your code on the provided Git repository
			\end{itemize}
            \item set up an appointment for discussing your project
        \end{enumerate}

        \vfill

        \item group projects are allowed (max 4 persons)
        %
        \begin{itemize}
            \item rule of thumb: $\sim90$ working hours per person per project
        \end{itemize}
    \end{itemize}
\end{frame}

\begin{frame}[c, allowframebreaks]{Sorts of Projects}

    \begin{block}{Classic \courseAcronym{} Project}
        Design and develop a software artifact leveraging upon MAS or LP
        %
        \begin{itemize}
            \item[eg] simulation, application, video-game, etc.
            \item Workflow:
            \vspace{-10pt}
            \begin{multicols}{2}
				\begin{enumerate}
					\item Sketch the idea
					\item Design
					\item Write \alert{automated} tests
					\item Implement
					\item Pack / deploy for reusability
				\end{enumerate}
			\end{multicols}
            \item Previous projects in this category: \ccite{AgentRace,BacterialAgent,PokerBraggion}
            \item[!] Innovation is limited here, thus we expect
            %
            \begin{itemize}
                \item design to be optimal
				\item implementation to be complete
				\item testing to be pervasive
            \end{itemize}
        \end{itemize}
    \end{block}

    \begin{block}{Advanced \courseAcronym{} Project}
        Provide an implementation for some \alert{scientifically-relevant} agent or logic framework \alert{from the literature}
        %
        \begin{itemize}
            \item[eg] agent-based simulation/programming tool, logic reasoner, etc. 
            \item Workflow:
            %
            \begin{enumerate}
				\item Select \& study a paper form the literature
				\item Design your implementation for the proposed contribution
				\item Write \alert{automated} tests
				\item Prototype an implementation
				\item Pack / deploy for a demo
			\end{enumerate}
            \item Previous projects in this category: \ccite{LidarBosello,RetineuraliSabbatini,AbudtivemetaintDisanti}
            \item[!] Innovation is high here, thus we
            %
            \begin{itemize}
                \item we mostly care about the proposed design
                \item implementation can be sub-optimal or embryonic
            \end{itemize}
        \end{itemize}
    \end{block}

    \begin{block}{Research \courseAcronym{} Project}
        Help us extending some research product of ours with some feature
        %
        \begin{itemize}
            \item[eg] TuSoW\ccite{tusow-icccn2019}, LPaaS\ccite{lpaas-bdcc2}, tuProlog\ccite{2pkt-swx16}
            \item Workflow:
            %
            \begin{enumerate}
                \item Study the target research product
                \item Collect requirements by interacting with us
                \item Co-design the feature by interacting with us
                \item Implement \& test the feature
            \end{enumerate}
            \item Previous projects in this category: \ccite{ClausesBonarrigo,GalassiKotlinbdi}
            \item[!] Innovation is high here, but the activity is very constrained
            %
            \begin{itemize}
                \item design and implementation must be expert-level
                \item testing, and continuos integration are mandatory
                \item you will be considered part of development team
                \item you will be part of the author list of any subsequent paper
            \end{itemize}
        \end{itemize}
    \end{block}

    \begin{block}{Systematic Literature Review (SLR)}
        Produce a scientific report on a particular research goal of choice (related to the topics of the course)
        %
        \begin{itemize}
            \item Workflow will be explained by Prof. Omicini via an ad-hoc lecture
            \item Previous projects in this category: \ccite{SlrParallelLp,SKEMormoneAspeeEncina,SymbolicBacchiani}
            \item[!] Focus here is on bibliographic exploration(s)
            %
            \begin{itemize}
                \item reproducibility of the exploration is essential
                \item soundness and clearness of conclusions is of paramount importance
                \item may easily become worth of actual publication!
            \end{itemize}
        \end{itemize}
    \end{block}
\end{frame}

%===============================================================================
\section*{}
%===============================================================================

%/////////
\frame{\titlepage}
%/////////

%===============================================================================
\section*{\refname}
%===============================================================================

%%%%
\setbeamertemplate{page number in head/foot}{}
%/////////
% \begin{frame}[c,noframenumbering]{\refname}
\begin{frame}[t,allowframebreaks,noframenumbering]{\refname}
%	\tiny
	\scriptsize
%	\footnotesize
	\bibliographystyle{apalike-AMS}
	\bibliography{ise-lab-about}
\end{frame}
%/////////

%%%%%%%%%%%%%%%%%%%%%%%%%%%%%%%%%%%%%%%%%%%%%%%%%%%%%%%%%%%%%%%%%%%%%%%%%%%%%%%%
\end{document}
%%%%%%%%%%%%%%%%%%%%%%%%%%%%%%%%%%%%%%%%%%%%%%%%%%%%%%%%%%%%%%%%%%%%%%%%%%%%%%%%
